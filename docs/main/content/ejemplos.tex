\chapter{Ejemplos}
    
    \section{Jerarquía de títulos: Section}
        \subsection{Subsection}
            \subsubsection{Subsubsection}
                \paragraph{Paragraph}
                    \subparagraph{Subparagraph}
                    
    \section{Cita}
        En esta oración se hace referencia a dos trabajos \cite{knn_franco_franz,herman}. Para esto se utilizo el codigo: 
        \texttt{\string\cite\string{knn\_franco\_franz,herman\string}}
        %\texttt{\\cite\{knn_franco_franz,herman\}}.
        
    \section{Referencia}
        \subsection{Crear un punto referenciable}\label{subsec:referenciable}
            Esta subseccion es referenciable, para lograrlo, se agrego \texttt{\\label{subsec:referenciable}}. Donde \texttt{subsec:referenciable} es el nombre de la etiqueta, el prefijo \texttt{subsec} corresponde a el tipo de elemento referenciado (esta es una convencion, su uso no es obligatorio \cite{latex-labels}).
            
        \subsection{Referenciar}
            En esta subseccion se hace referencia a la seccion \nameref{subsec:referenciable}.
            
            \begin{itemize}
                \item Solo numero: \ref{subsec:referenciable}
                \item Solo nombre: \nameref{subsec:referenciable}
                \item Numero y nombre: \fullref{subsec:referenciable}
                \item Numero y nombre \textit{custom}: \fullref[Nombre \textit{Custom}]{subsec:referenciable}
            \end{itemize}
        
    \section{Texto entre comillas}
        Para que el texto figure entre `comillas' se deben utilizar los caracteres \char18\string' de forma similar a este ejemplo \texttt{\char18comillas\string'}.

    \section{Ecuaciones}
        \subsection{Normal}
        \begin{equation}
                {\frac {\sum \limits _{i=1}^{n}w_{i}x_{i}}{\sum \limits _{i=1}^{n}w_{i}}}
        \end{equation}
        
        \subsection{Grande}
        \begin{equationBig}
                {\frac {\sum \limits _{i=1}^{n}w_{i}x_{i}}{\sum \limits _{i=1}^{n}w_{i}}}
        \end{equationBig}
        
    \section{Listados}
        \subsection{Enumeracion}
        
            \begin{enumerate}
                \item Primer punto
                \item Segundo punto
                \item Tercer punto
            \end{enumerate}
            
        \subsection{Items}
        
            \begin{itemize}
                \item Primer punto
                \item Segundo punto
                \item Tercer punto
            \end{itemize}
        
        \subsection{Descripcion}
        
            \begin{description}
                \item[Primer punto] Primer punto
                \item[Segundo punto] Segundo punto
                \item[Tercer punto] \nextLine
                    Tercer punto
            \end{description}
        
        \subsection{Múltiples niveles}
        
            \begin{itemize}
                \item Primer punto
                \begin{enumerate}
                    \item Primer punto
                    \item Segundo punto
                    \item Tercer punto
                \end{enumerate}
                
                \item Segundo punto
                \begin{itemize}
                    \item Primer punto
                    \item Segundo punto
                    \item Tercer punto
                \end{itemize}
                \item Tercer punto
                \begin{description}
                    \item[Primer punto] Primer punto
                    \item[Segundo punto] Segundo punto
                    \item[Tercer punto] \nextLine 
                        Tercer punto
                \end{description}
            \end{itemize}