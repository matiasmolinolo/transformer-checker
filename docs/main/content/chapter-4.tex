\chapter{Trainability of Transformers}

Having already analyzed the Transformer architecture, we will now proceed with a discussion on whether these architectures are
capable of being trained to recognise context-free grammars.

However, before we dive into this analysis on the \emph{trainability} of Transformers, which will be backed by experimental data, we
first must analyze whether these architectures are capable of learning context-free grammars or not - in essence,
are they \emph{expressive} enough to generate an internal model of a context-free grammar?

Bhattamistra et. al. \cite{bhattamistra-transformers-formal-langs} prove that these architectures are expressive enough to recognize
the Shuffle-Dyck-$k$ family of languages and notes the special case of $k=1$, where Shuffle-Dyck-$k$ is equal to Dyck-1 languages.
More specifically, Ströbl et. al. \cite{strobl2024formal} specifies that this Transformer is one with soft-attention
(i.e.: softmax attention, as defined in \ref{chapter:3}) with future masking, positional encoding only, no layer normalization and no residual connections.

In our experiments, we found that this Transformer may be expressive enough to recognize this family of languages, but is not
trainable enough, however, a Transformer with soft-attention with future masking, positional encoding only, layer normalization and residual connections
is trainable enough to recognize and generalize to both the training and test datasets, reaching 100\% accuracy and a loss of 0 in both sets.
